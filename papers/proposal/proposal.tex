\documentclass[a4paper, 12pt]{article}
\usepackage{dirtree}
%\usepackage{cite}
\usepackage[utf8]{inputenc}
\usepackage{listings}
\usepackage{bm}
\usepackage{pdfpages}
\usepackage{fancyhdr}
\usepackage[colorlinks, citecolor=black, urlcolor=blue, bookmarks=false, hypertexnames=true]{hyperref}
\usepackage{biblatex}
\addbibresource{bibilo.bib}
\usepackage{tocloft}
%\renewcommand{\cftsecleader}{\cftdotfill{\cftdotsep}}
%\renewcommand{\cftsecpagefont}{}% Remove \bfseries from section titles' page in ToC
\usepackage{verbatim}
\usepackage{amsmath}
\usepackage{amssymb}
\usepackage[utf8]{inputenc}
%\newcommand\logoPath{TU-logo.png}
%\newcommand\logoFileName{TU-logo}
%\newcommand\logoScaleFactor{0.7}
\usepackage{graphicx}
\usepackage{setspace}
\usepackage{afterpage}
\usepackage{tikz}
\usepackage{amsthm}
\usetikzlibrary{shapes,arrows, positioning}
\usepackage{abstract}
\renewcommand{\abstractnamefont}{\normalfont\large\bfseries}
\newcommand\PROJECTNAME{Geocold}
%\hypersetup{linkbordercolor=black}

\newtheorem{definition}{Definition}


\begin{document}
\graphicspath{\logoPath}
\renewcommand*\contentsname{Table Of Contents}
\newcommand\myemptypage{
		\null
		\thispagestyle{empty}
		\addtocounter{page}{-1}
		\newpage
}

\lstset{language=C++,
		tabsize=3,
				frame=tb,
                basicstyle=\ttfamily,
                keywordstyle=\color{blue}\ttfamily,
                stringstyle=\color{red}\ttfamily,
                commentstyle=\color{green}\ttfamily,
                morecomment=[l][\color{magenta}]{\#}
}
%\lstdefinestyle{mystyle}{frame=tb,
%						language=C++,
%						tabsize=3,
%						numbersep=5pt}

%\lstset{style=mystyle}
\begin{titlepage}
  \begin{center}
      \textbf{
      	  \begin{huge}Project Proposal:\\ \end{huge}
      	  \begin{huge}
      	  \PROJECTNAME \\
			Ray Tracer (Differentiable?) in C++
			\end{huge}}\\
    \vspace{0.5\baselineskip}
    {\Large Authors:  Prakash Chaulagain, Nishar Arjyal, and Pramish Paudel}\\
   	\Large{Roll Numbers: 076BCT045, 076BCT042, 076BCT047} \\
   	\vspace{0.5\baselineskip}
    \centering
      Submitted to the Department of
      Electronics and Computer Engineering
      in Partial Fulfillment of the Requirements of the 3rd Year \textbf{Computer Graphics} Course  
    at 	\\
    {\Large Pulchowk Campus }\\
    {\Large IOE, Tribhuwan University}\\
    \vspace{0.3\baselineskip}
    \today\\
  \end{center}
  \vspace{2\baselineskip}
  {
  \raggedright
		Accepted by: \dotfill

  \raggedleft
  \vspace{1\baselineskip}
  Mr. Basanta Joshi\\
  Assistant Professor at \\
  The Department of Electronics and Computer Engineering\\
  }
  \vspace{2\baselineskip}	
	 Date of Submission: \today \\
  	Expected Date of Completion: July, 2022 
\end{titlepage}
\newpage
\begin{center}
	{\large \textbf{\PROJECTNAME}}\\
	by
	{Prakash Chaulagain, Nishar Arjyal, and Pramish Paudel}\\
	\vspace{0.2\baselineskip}
	\begin{spacing}{0.8}
	{Submitted to the Department of Electronics and Computer Engineering \\
	on \today \\ in Partial Fulfillment of the Requirements for the 3rd Year 
	Computer Graphics Course in Computer Engineering}\\
	\end{spacing}
\end{center}
\begin{abstract}
	Ray tracing has a rich history in the history of computing and computer 
	graphics. With this manuscript, we propose to build an offline ray tracing software using 
	the Vulkan graphics/compute library in C++. Our renderer is supposed to 
	work generically, as in take as input any file containing geometric data,
	perform a mesh render pass in order to render a mesh of the described 
	scene and then perform ray tracing with a separate pass. In this paper, we cover 
	the mathematical principles that we follow as we build our ray tracing software.
\end{abstract}

\newpage

\tableofcontents
\newpage

\begin{section}{Acknowledgement}
	Our project idea is the product of excellent supervision of all of our 
	instructors, most notably our lecturer Mr. Basanta Joshi. We could not 
	have been able to develop interest in computer graphics as a field of study 
	without the constant inspiration provided to us by all of our lecturers and 
	lab instructors and assistants. Some of the credit should also go to 
	the college administration for their efforts in the smooth functioning 
	of all of our classes and labs safely and securely despite the unprecedented 
	times of the pandemic. 
\end{section}

\newpage

\newpage
\pagestyle{fancy}
\fancyhead[C]{\PROJECTNAME}
\fancyhead[L]{}

\renewcommand{\headrulewidth}{0pt}
\renewcommand{\footrulewidth}{0pt}

\section{Objectives}
\begin{itemize}
	\item To understand the graphics pipeline.
	\item To become familiar with modern GPU architectures.
	\item To become familiar with GPU programming models and GPU computing (massively parallel computing).
	\item To understand and uncover existing ray tracing techniques.
	\item To gain a degree of familiarity with common graphics and GPU compute APIs, 
	and understand their abstraction mechanisms.
\end{itemize}

\section{Introduction}
	

Rendering is the task of taking a scene composed of many geometric 
objects arranged in 3D space and computing a 2D image that shows the object 
as viewed from a particular viewpoint. The goal of our project \PROJECTNAME~is to create 
a photorealistic renderer by implementing a .obj file loader which then 
creates a mesh of our scene, then finally we implement a ray tracer 
which will correctly color every object in the scene. 
Over the next few sections, we will try and set the mathematical 
basis/principles used in our project and the API that we have attempted to 
design based on those principles.

\section{Primitives}
In order to explain the way our ray tracer works, we need to 
set the ground with some basic mathematical 
terms and notations. This will not only give us a more 
technical basis for coming up with a good, mathematically 
correct API, but also help someone using the same 
piece of code to connect the pieces together. The next few of definitions are based on 
Farin(\cite{farin}) and Goldman(\cite{Goldman1985IllicitEI}).

\begin{definition}[Affine Combination in 3D]
	An affine combination of vectors $\bm{x_{1}}, \bm{x_{2}}, \bm{x_{3}}$, 
	$\bm{x_{i}}\in \mathbb{R}^{3}$
	is the linear combination $\sum_{i=1}^{3}\lambda_{j}\bm{x_{i}}$ such that
	$\sum_{j=1}^{3}\lambda_{j}=1$
\end{definition}

\begin{definition}[Affine Map]
	$\Phi:\mathbb{R}^{3}\to \mathbb{R}^{3}$ is an affine map if it leaves affine combinations
	invariant. That is, if
	$$
		\bm{a} = \sum_{j=1}^{3}\lambda_{j}\bm{x_{j}}, \sum_{j=1}^{3}\lambda_{j}=1, \bm{a}, x_{j}\in \mathbb{R}^{3}
	$$
	then, 
	\begin{equation}\label{eqn:affinecomb}
		\Phi(\bm{a}) = \sum_{j=1}^{3}\lambda_{j}\Phi(x_{j}); \Phi(x_{j}), \Phi(\bm{a}) \in \mathbb{R}^{3}.
	\end{equation}
\end{definition}

The  equation\eqref{eqn:affinecomb} specifies the correct way of weighing the 
points $x_{j}$ such that their weighted average is the point $\bm{a}$.

In a given coordinate system, a point $\bm{x}$ is represented in the form of a coordinate triple, 
which we also denote by $\bm{x}$. An affine map now takes the form 
\begin{equation}\label{eqn:affinefunction}
\Phi(\bm{x}) = A\bm{x} + \bm{v}
\end{equation}
The proof that equation\eqref{eqn:affinefunction} is affine is in Farin\cite{farin}.

\subsection{On Points and Vectors}
Goldman (see \cite{Goldman1985IllicitEI}) describes how points and vectors 
are different and ought to be treated differently. Their work also describes 
the way of dealing with their differences.

Points have positions but not direction or length, while vectors have a direction 
and a length. Not all operations that can be applied for vectors can be applied for 
points. Points cannot be added, however addition like operations 
such as an affine combination (also called \textit{barycentric combination}). Although 
we will not go into a comprehensive list of operations that can be 
applied for points and for vectors, we will list (table:\ref{table:pointvec}) out some of the common
operations that are commonly implemented in any legitimate ray tracing API 
(like PBRT\cite{pbrt} and the ones mentioned in Nvidia's Ray Tracing Gems\cite{Haines2019}). 
We denote points with capital letters like $P$ or $Q$, and we denote vectors 
with small boldface letters like $\bm{u}$ or $\bm{v}$.

\setlength{\tabcolsep}{2em}
\begin{table}[htb]
	\centering
	\begin{tabular}{|c c c|}\hline
	Operation & Legal & Undefined	\\ \hline
	Addition & $\bm{u}$+$\bm{v}$(v), P+$\bm{u}$(p) & P+Q \\
	Subtraction & $\bm{u}$-$\bm{v}$(v), P-$\bm{u}$ & $\bm{u}-\bm{P}$\\
	Scalar Multiplication & c.$\bm{v}$(v) & c.P \\
	Dot Product & $\bm{u}.\bm{v}$(scalar) & P.$\bm{u}$, P.Q \\
	Cross Product & $\bm{u}\times \bm{v}$(v) & P $\times$Q, P $\times\bm{u}$\\ \hline
	\end{tabular}
	\caption{Common operations defined for points and vectors. (v in the braces 
	represents vector and p represents point)}
	\label{table:pointvec}
\end{table}

In C++, we can describe this behavior difference simply through the type system to 
create an API that would at the least follow basic axioms in mathematics and algebra.

To come up with a decent and mathematically correct ray tracer, it is essential 
that we use the C++ type system to handle basic principles of algebra, and analysis.

\begin{lstlisting}[caption=Sample code for Point3 and Vec3 types]
	template<typename T>
	struct Vec3; //forward declaration

	template <typename T>
	struct Point3 {
		T x_;
		T y_;
		T z_;
		Point3(T x, T y, T z) noexcept 
		   :x_{x}, 
			y_{y}, 
			z_{z} {}
		
		Point3 operator+(const Vec3<T>& vec) noexcept {
			return Point3<T>{
				x_ + vec.x_,
				y_ + vec.y_,
				z_ + vec.z_
			};
		} //this implements Point3 + Vec3

		Vec3<T> operator-(const Point3& rhs) noexcept {
			return Vec3<T>{
				x_ - rhs.x_,
				y_ - rhs.y_,
				z_ - rhs.z_
			};
		} // Point3 - Point3 -> Vec3 

		Point3<T> operator-(const Vec3<T>& vec) noexcept {
			return Point3<T>{
				x_ - vec.x_,
				y_ - vec.y_,
				z_ - vec.z_
			};
		}
	};

	template <typename T>
	struct Vec3<T> {
		T x_;	
		T y_;	
		T z_;	

		Vec3(T x, T y, T z) noexcept 
			:x_{x}
			,y_{y}
			,z_{z} {}
		
		Vec3<T> operator-() const {
			return Vec3<T>{
				-x_,
				-y_, 
				-z_
			};
		}

		Vec3<T> operator*(T scalar) {
			return Vec3<T>{
				x*scalar,
				y*scalar,
				z*scalar
			};
		}

//other trivial operations like 
//vector addition, 
//subtraction, dot product and cross 
//product are also implemented.
//Vec3 - Point3 is not implemented 
//as it is an invalid operation.
	};
\end{lstlisting}

\subsection{Other Primitives}
\begin{definition}[Normal]
	A surface normal is a vector perpendicular to a surface at a 
	particular position.
\end{definition}

It is necessary to have a separate type for normals as 
they are treated differently to vectors (normals take 
object properties into account). Advanced ray tracers 
(like \cite{pbrt}) also take into account the various
partial derivatives of the surface normal with respect
to the local parametrization of the curve. In 
ray tracing, it is essential to know the normal 
to the surface at which the ray-triangle 
intersection test is satisfied so as to 
compute the direction in which the ray will be reflected next.

\begin{definition}[Ray]
	We represent a ray as a parametric line in \PROJECTNAME.
	If $\bm{a}$ and $\bm{b}$ are two points such that 
	the ray starts at $\bm{a}$ then, the line between 
	$\bm{a}$ and $\bm{b}$ is given by the parametric equation: 
	\begin{equation}
		\bm{r}(t) = (1-t)\bm{a} + t\bm{b}, t\in [0,1]
	\end{equation}
\end{definition}
	This is obviously an affine combination of the points $\bm{a}$ and $\bm{b}$.
	However, on slight modification, we can implement a ray 
	as a linear combination of a point and a vector as such:
	\begin{equation}
		\bm{r}(t) = \bm{a} + t(\bm{b}-\bm{a})
	\end{equation}

	But, we need the parameter $t$ to be in a set interval so
	that we won't have to track the ray farther than we need to
	(upto when the ray passes the ray-triangle intersection test).
	The ray can also not go behind the camera or eye.
	So, we need fields $tmin$ and $tmax$ for our struct.

	\begin{lstlisting}
		struct Ray {
			Point3<float> a;
			float tmin;
			Vec3<T> direction; //b-a
			float tmax;
		}; //ray is left unparametrized
	\end{lstlisting}

	Since most computer graphics projects
	define common camera models, we won't go
	to depth on the camera model for ray tracing in 
	this paper. 

	We will finally end this section with a few words 
	on radiosity and light model.

\subsection{Raidosity and Light Models}
\begin{center}
\begin{figure}
	\includegraphics[width=0.6\linewidth, scale=0.8]{lightem.png}
	\centering
	\caption{Light as an electromagnetic wave.}
\end{figure}
\end{center}
Light is an electromagnetic wave that propagates through space. 
A convincing computer model for light would incorporate into it 
the direction of propagation, speed, wave-lenth, amplitude, polarization 
and so on. Although not all of these parameters are incorporated in 
standard ray tracers, ours will be a lightweight one in which we 
shall model light without any polarizability.
Our ray tracer however, should be able to handle the effect of 
the wavelength of light in the ways it behaves after striking 
a surface/object in the scene. Our 
ray tracer will disregard the spectrums and 
only deal with RGB. First, let's define some terms from radiometry.

\begin{definition}[Radiant Energy]
Electromagnetic radiations carry energy through space.
The energy carried by a single photon emitted by a source 
of illumination is given by 
$$
	Q = \cfrac{hc}{\lambda}
$$
\end{definition}

\begin{definition}[Radiant Flux]
Radiant Flux or power is defined as the total amount of energy passing 
through a surface or region of space per unit time.
$$
	\Phi = \cfrac{dQ}{dt}
$$
\end{definition}

\begin{definition}[Radiant Flux Density]
	Radiant Flux Density is the radiant flux per unit area at a 
	point on a surface, where the surface could be real or imaginary.
\end{definition}

\begin{definition}[Irradiance E]
	The radiant flux arriving at a surface per unit area 
	is called irradiance. 
	$$
		E = \cfrac{d\Phi}{dA}
	$$
\end{definition}

\begin{definition}[Radiant Exitance M]
	The radiant flux leaving from a surface per unit area 
	is called radiant exitance. 
	$$
		M =  \cfrac{d\Phi}{dA}
	$$
\end{definition}
\begin{center}
\begin{figure}
	\includegraphics[width=0.8\linewidth, scale=0.6]{raynormal.png}
	\centering
	\caption{Ray of Light Intersecting a surface}
	\label{figure:rayandnormal}
\end{figure}
\end{center}

\begin{definition}[Radiance]
	Imagine an elemental cone to represent a 
	ray that intersects a surface making an 
	angle $\theta$ with the surface normal as in figure~\ref{figure:rayandnormal}.
	The projected area of the ray-surface intersection area $dA$
	is $dAcos\theta$. The radiance is the radiant flux 
	density per unit elemental cone or per unit solid 
	angle $d\omega$. 
	$$
		L = \cfrac{d^{2} \Phi}{dA cos\theta d\omega}
	$$
\end{definition}
In order to avoid the complexity of having to take limits to 
deal with the radiance at above and below a surface, we define radiance 
arriving at a point on a surface and the radiance leaving that point. 

Consider a point p on the surface of an object. Let $L_{i}(\bm{p}, \bm{\omega})$


\newpage
\printbibliography


\end{document}


