\documentclass[12pt]{article}
\usepackage{dirtree}
%\usepackage{cite}
\usepackage[utf8]{inputenc}
\usepackage{listings}
\usepackage{pdfpages}
\usepackage{fancyhdr}
\usepackage{biblatex}
\addbibresource{bibilo.bib}
\usepackage{tocloft}
%\renewcommand{\cftsecleader}{\cftdotfill{\cftdotsep}}
%\renewcommand{\cftsecpagefont}{}% Remove \bfseries from section titles' page in ToC
\usepackage{verbatim}
\usepackage{amsmath}
\usepackage[utf8]{inputenc}
%\newcommand\logoPath{TU-logo.png}
%\newcommand\logoFileName{TU-logo}
%\newcommand\logoScaleFactor{0.7}
\usepackage{graphicx}
\usepackage{setspace}
\usepackage{afterpage}
\usepackage{tikz}
\usetikzlibrary{shapes,arrows, positioning}
\usepackage{abstract}
\renewcommand{\abstractnamefont}{\normalfont\large\bfseries}
\newcommand\PROJECTNAME{Geocold}


\begin{document}
\graphicspath{\logoPath}
\renewcommand*\contentsname{Table Of Contents}
\newcommand\myemptypage{
		\null
		\thispagestyle{empty}
		\addtocounter{page}{-1}
		\newpage
}
\lstdefinestyle{mystyle}{frame=tb,
						%language=C++,
						tabsize=3,
						numbersep=5pt}
\lstset{style=mystyle}
\begin{titlepage}
  \begin{center}
      \textbf{
      	  \begin{huge}Project Proposal:\\ \end{huge}
      	  \begin{huge}
      	  \PROJECTNAME \\
			Ray Tracer (Differentiable?) in C++
			\end{huge}}\\
    \vspace{0.5\baselineskip}
    {\Large Authors:  Prakash Chaulagain, Nishar Arjyal, and Pramish Paudel}\\
   	\Large{Roll Numbers: 076BCT045, 076BCT042, 076BCT047} \\
   	\vspace{0.5\baselineskip}
    \centering
      Submitted to the Department of
      Electronics and Computer Engineering
      in Partial Fulfillment of the Requirements of the 3rd Year \textbf{Computer Graphics} Course  
    at 	\\
    {\Large Pulchowk Campus }\\
    {\Large IOE, Tribhuwan University}\\
    \vspace{0.3\baselineskip}
    \today\\
  \end{center}
  \vspace{2\baselineskip}
  {
  \raggedright
		Accepted by: \dotfill

  \raggedleft
  \vspace{1\baselineskip}
  Mr. Basanta Joshi\\
  Assistant Professor at \\
  The Department of Electronics and Computer Engineering\\
  }
  \vspace{2\baselineskip}	
	 Date of Submission: \today \\
  	Expected Date of Completion: July, 2022 
\end{titlepage}
\myemptypage
\begin{center}
	{\large \textbf{\PROJECTNAME}}\\
	by
	{Prakash Chaulagain, Nishar Arjyal, and Pramish Paudel}\\
	\vspace{0.2\baselineskip}
	\begin{spacing}{0.8}
	{Submitted to the Department of Electronics and Computer Engineering \\
	on \today \\ in Partial Fulfillment of the Requirements for the 3rd Year 
	Computer Graphics Course in Computer Engineering}\\
	\end{spacing}
\end{center}
\begin{abstract}
	Ray tracing has a rich history in the history of computing and computer 
	graphics. With this manuscript, we propose to build an offline ray tracing software using 
	the Vulkan graphics/compute library in C++. Our renderer is supposed to 
	work generically, as in take as input any file containing geometric data,
	perform a mesh render pass in order to render a mesh of the described 
	scene and then perform ray tracing with a separate pass. In this paper, we cover 
	the mathematical principles that we follow as we build our ray tracing software.
\end{abstract}

\newpage

\tableofcontents
\newpage

\begin{section}{Acknowledgement}
	Our project idea is the product of excellent supervision of all of our 
	instructors, most notably our lecturer Mr. Basanta Joshi. We could not 
	have been able to develop interest in computer graphics as a field of study 
	without the constant inspiration provided to us by all of our lecturers and 
	lab instructors and assistants. Some of the credit should also go to 
	the college administration for their efforts in the smooth functioning 
	of all of our classes and labs safely and securely despite the unprecedented 
	times of the pandemic. 
\end{section}

\newpage

\newpage
\pagestyle{fancy}
\fancyhead[C]{\PROJECTNAME}
\fancyhead[L]{}

\renewcommand{\headrulewidth}{0pt}
\renewcommand{\footrulewidth}{0pt}

\section{Objectives}
\begin{itemize}
	\item To understand the graphics pipeline.
	\item To become familiar with modern GPU architectures.
	\item To become familiar with GPU programming models and GPU computing (massively parallel computing).
	\item To understand and uncover existing ray tracing techniques.
	\item To gain a degree of familiarity with common graphics and GPU compute APIs, 
	and understand their abstraction mechanisms.
\end{itemize}

\section{Introduction}
Rendering is the task of taking a scene composed of many geometric 
objects arranged in 3D space and computing a 2D image that shows the object 
as viewed from a particular viewpoint. The goal of our project \PROJECTNAME 
is to implement an obj file loader which then creates a mesh of our scene,
then finally we implement a ray tracer which will correctly color every
object in the scene. Over the next few sections, we will try and set the mathematical 
basis/principles used in our project and the API that we have attempted to design based on those 
principles.

\section{Mathematical Basis}





\subsection{On Basic Geometric Primitives}

\section{Existing Systems}


\newpage
The package is suppposed to have the following directory structure:
\\
\dirtree{%
	.1 Geocold.
	.2 papers.
	.2 src.  
	.2 testbench.  
	.2 external.
} 


\subsection{System Block Diagram}
The following is a block diagram of our system:

\newpage

\section{Project Scope}


\section{Project Schedule}


\newpage
\printbibliography


\end{document}


